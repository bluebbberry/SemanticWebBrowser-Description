\documentclass[12pt,a4paper]{article}

% German language support
\usepackage[ngerman]{babel}
\usepackage[utf8]{inputenc}
\usepackage[T1]{fontenc}

% Additional packages
\usepackage{amsmath}
\usepackage{graphicx}
\usepackage{hyperref}
\usepackage{geometry}
\usepackage{fancyhdr}

% Page layout
\geometry{margin=2.5cm}
\pagestyle{fancy}
\fancyhf{}
\fancyhead[L]{Beispieldokument}
\fancyhead[R]{\today}
\fancyfoot[C]{\thepage}

% Document information
\title{SemanticWebBrowser: Architectural description of a user-friendly, deterministic, language-interface-based, web-paradigm-agnostic, IDE-like Web-Browser}
\author{Jan Bingemann}
\date{\today}

\begin{document}

\maketitle

\tableofcontents
\newpage

\section{Einleitung}

Dies ist ein Beispieldokument, das die Verwendung von LaTeX mit deutscher Sprache demonstriert. LaTeX ist ein leistungsfähiges Textsatzsystem, das besonders in der wissenschaftlichen Welt weit verbreitet ist.

\subsection{Vorteile von LaTeX}

LaTeX bietet viele Vorteile gegenüber herkömmlichen Textverarbeitungsprogrammen:

\begin{itemize}
    \item Professionelle Typographie
    \item Automatische Nummerierung von Kapiteln, Abschnitten und Formeln
    \item Einfache Verwaltung von Literaturverzeichnissen
    \item Plattformunabhängigkeit
    \item Kostenlos und open source
\end{itemize}

\section{Mathematische Formeln}

LaTeX eignet sich hervorragend für das Setzen mathematischer Formeln. Hier sind einige Beispiele:

\subsection{Inline-Mathematik}
Der Satz des Pythagoras besagt, dass $a^2 + b^2 = c^2$ für rechtwinklige Dreiecke gilt.

\subsection{Abgesetzte Formeln}
Die Euler'sche Identität ist eine der schönsten Formeln der Mathematik:
\begin{equation}
    e^{i\pi} + 1 = 0
\end{equation}

Die quadratische Formel lautet:
\begin{equation}
    x = \frac{-b \pm \sqrt{b^2 - 4ac}}{2a}
\end{equation}

\section{Deutsche Sonderzeichen}

LaTeX kann problemlos mit deutschen Umlauten und dem Eszett umgehen: ä, ö, ü, Ä, Ö, Ü, ß.

Auch Anführungszeichen werden korrekt dargestellt: "`Deutsche Anführungszeichen"' sehen so aus.

\section{Listen und Aufzählungen}

\subsection{Nummerierte Liste}
\begin{enumerate}
    \item Erster Punkt
    \item Zweiter Punkt
        \begin{enumerate}
            \item Unterpunkt a
            \item Unterpunkt b
        \end{enumerate}
    \item Dritter Punkt
\end{enumerate}

\subsection{Beschreibungsliste}
\begin{description}
    \item[LaTeX] Ein Textsatzsystem für wissenschaftliche Dokumente
    \item[PDF] Portable Document Format
    \item[UTF-8] Eine Zeichenkodierung für Unicode
\end{description}

\section{Tabellen}

\begin{table}[h!]
\centering
\caption{Beispieltabelle mit deutschen Inhalten}
\begin{tabular}{|l|c|r|}
\hline
\textbf{Name} & \textbf{Alter} & \textbf{Stadt} \\
\hline
Anna Müller & 25 & München \\
Klaus Schmidt & 30 & Berlin \\
Maria Weber & 22 & Hamburg \\
\hline
\end{tabular}
\label{tab:beispiel}
\end{table}

Wie in Tabelle \ref{tab:beispiel} zu sehen ist, können Tabellen einfach erstellt und referenziert werden.

\section{Zitate und Referenzen}

"`Die Grenzen meiner Sprache bedeuten die Grenzen meiner Welt."' -- Ludwig Wittgenstein

Für wissenschaftliche Arbeiten ist ein ordnungsgemäßes Zitieren unerlässlich.

\section{Fazit}

LaTeX ist ein mächtiges Werkzeug für die Erstellung professioneller Dokumente. Mit der richtigen Konfiguration für deutsche Texte lassen sich auch komplexe wissenschaftliche Arbeiten erstellen.

Weitere Informationen finden Sie in der umfangreichen LaTeX-Dokumentation oder in entsprechenden Büchern zum Thema.

\end{document}
